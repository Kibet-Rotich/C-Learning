\documentclass{article}
\usepackage{amsmath}
\usepackage{array}
\usepackage{booktabs}
\usepackage{tikz}
\usetikzlibrary{automata, positioning}

\title{LR(0) Analysis for the Given Grammar}
\author{}
\date{}

\begin{document}

\section*{Grammar}
Non-terminals: \( N = \{S, E, A\} \)

Terminals: \( T = \{\text{id}, ;, =, \$\} \) (where \$ is the end-marker)

Productions:
\begin{enumerate}
    \item \( S \to E \$ \)
    \item \( E \to E ; A \)
    \item \( E \to A ; \)
    \item \( A \to \text{id} \)
    \item \( A \to A = A \)
\end{enumerate}

\subsection*{a) LR(0) Item Sets and CFSM (DFA)}

The canonical collection of LR(0) item sets is:

\begin{align*}
I_0 &= \text{closure}(\{[S \to .E \$]\}) \\
    &= \left\{ 
    \begin{array}{l}
    S \to .E \$ \\
    E \to .E ; A \\
    E \to .A ; \\
    A \to .\text{id} \\
    A \to .A = A
    \end{array}
    \right\} \\[1em]
I_1 &= \text{goto}(I_0, E) = \left\{ 
    \begin{array}{l}
    S \to E.\$ \\
    E \to E.; A
    \end{array}
    \right\} \\[1em]
I_2 &= \text{goto}(I_0, A) = \left\{ 
    \begin{array}{l}
    E \to A.; \\
    A \to A.= A
    \end{array}
    \right\} \\[1em]
I_3 &= \text{goto}(I_0, \text{id}) = \{ A \to \text{id}. \} \\[1em]
I_4 &= \text{goto}(I_1, ;) = \left\{ 
    \begin{array}{l}
    E \to E;.A \\
    A \to .\text{id} \\
    A \to .A = A
    \end{array}
    \right\} \\[1em]
I_5 &= \text{goto}(I_2, ;) = \{ E \to A;. \} \\[1em]
I_6 &= \text{goto}(I_2, =) = \left\{ 
    \begin{array}{l}
    A \to A=.A \\
    A \to .\text{id} \\
    A \to .A = A
    \end{array}
    \right\} \\[1em]
I_7 &= \text{goto}(I_4, A) = \left\{ 
    \begin{array}{l}
    E \to E;A. \\
    A \to A.= A
    \end{array}
    \right\} \\[1em]
I_8 &= \text{goto}(I_6, A) = \left\{ 
    \begin{array}{l}
    A \to A=A. \\
    A \to A.= A
    \end{array}
    \right\} \\[1em]
I_9 &= \text{goto}(I_1, \$) = \{ S \to E\$. \}
\end{align*}

Other important \texttt{goto} steps (for completeness):
\begin{itemize}
    \item \texttt{goto}(I_4, \text{id}) = I_3
    \item \texttt{goto}(I_6, \text{id}) = I_3
    \item \texttt{goto}(I_7, =) = I_6
    \item \texttt{goto}(I_8, =) = I_6
\end{itemize}

\textbf{CFSM Diagram (textual representation):}
\begin{verbatim}
I0 ──E──> I1 ──$──> I9 (accept state)
 │
 ├──A──> I2 ──;──> I5 (reduce 3)
 │      │
 │      └──=──> I6 ──id──> I3 (reduce 4)
 │             │
 │             └──A──> I8
 │
 ├──id──> I3
 │
 └──(via I4 and I6) other paths
\end{verbatim}

(Tikz drawing would look like a standard DFA graph with 10 states and the transitions listed above.)

\subsection*{b) LR(0) Parsing Table}

States are numbered 0 to 9 corresponding to \(I_0\)--\(I_9\).

\begin{center}
\begin{tabular}{c|cccc|ccc}
\toprule
State & \multicolumn{4}{c|}{Action} & \multicolumn{3}{c}{Goto} \\
\cmidrule(lr){2-5} \cmidrule(lr){6-8}
 & id & ; & = & \$ & S & E & A \\
\midrule
0 & s3 &   &   &   &   & 1 & 2 \\
1 &    & s4&   & s9&   &   &   \\
2 &    & s5& s6&   &   &   &   \\
3 & r4 & r4& r4& r4&   &   &   \\
4 & s3 &   &   &   &   &   & 7 \\
5 & r3 & r3& r3& r3&   &   &   \\
6 & s3 &   &   &   &   &   & 8 \\
7 & r2 & r2& s6/r2 & r2 &   &   &   \\
8 & r5 & r5& s6/r5 & r5 &   &   &   \\
9 &    &   &   & acc&   &   &   \\
\bottomrule
\end{tabular}
\end{center}

Notation:
\begin{itemize}
    \item \texttt{sN} = shift to state N
    \item \texttt{rN} = reduce by production N
    \item \texttt{acc} = accept
    \item blank = error
\end{itemize}

\subsection*{c) Nature of Conflicts}

There are \textbf{shift/reduce conflicts} on terminal \texttt{=} in:
\begin{itemize}
    \item State 7 (\(I_7\)): reduce by (2) \(E \to E ; A\) vs shift to \(I_6\)
    \item State 8 (\(I_8\)): reduce by (5) \(A \to A = A\) vs shift to \(I_6\)
\end{itemize}

These conflicts arise because:
\begin{itemize}
    \item The grammar is ambiguous (the production \(A \to A = A\) does not specify associativity).
    \item LR(0) has no lookahead, so it cannot decide whether to reduce the left-recursive assignment or continue shifting for another \texttt{=}.
    \item The language requires operator precedence/associativity (e.g., \texttt{=} should be right-associative) which LR(0) cannot resolve.
\end{itemize}

The grammar is \textbf{not LR(0)}.

\end{document}
